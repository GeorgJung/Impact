% !TEX TS-program = pdflatex
% !TEX encoding = UTF-8 Unicode

% This is a simple template for a LaTeX document using the "article" class.
% See "book", "report", "letter" for other types of document.

\documentclass[11pt]{article} % use larger type; default would be 10pt

\usepackage[utf8]{inputenc} % set input encoding (not needed with XeLaTeX)

%%% Examples of Article customizations
% These packages are optional, depending whether you want the features they provide.
% See the LaTeX Companion or other references for full information.

%%% PAGE DIMENSIONS
\usepackage{geometry} % to change the page dimensions
\geometry{a4paper} % or letterpaper (US) or a5paper or....
% \geometry{margins=2in} % for example, change the margins to 2 inches all round
% \geometry{landscape} % set up the page for landscape
%   read geometry.pdf for detailed page layout information

\usepackage{graphicx} % support the \includegraphics command and options

% \usepackage[parfill]{parskip} % Activate to begin paragraphs with an empty line rather than an indent

%%% PACKAGES
\usepackage{booktabs} % for much better looking tables
\usepackage{array} % for better arrays (eg matrices) in maths
\usepackage{paralist} % very flexible & customisable lists (eg. enumerate/itemize, etc.)
\usepackage{verbatim} % adds environment for commenting out blocks of text & for better verbatim
\usepackage{subfig} % make it possible to include more than one captioned figure/table in a single float
% These packages are all incorporated in the memoir class to one degree or another...

%%% HEADERS & FOOTERS
%\usepackage{fancyhdr} % This should be set AFTER setting up the page geometry
\pagestyle{empty} % options: empty , plain , fancy
% \renewcommand{\headrulewidth}{0pt} % customise the layout...
% \lhead{}\chead{}\rhead{}
% \lfoot{}\cfoot{\thepage}\rfoot{}

%%% SECTION TITLE APPEARANCE
\usepackage{sectsty}
\allsectionsfont{\sffamily\mdseries\upshape} % (See the fntguide.pdf for font help)
% (This matches ConTeXt defaults)

%%% ToC (table of contents) APPEARANCE
\usepackage[nottoc,notlof,notlot]{tocbibind} % Put the bibliography in the ToC
\usepackage[titles,subfigure]{tocloft} % Alter the style of the Table of Contents
\renewcommand{\cftsecfont}{\rmfamily\mdseries\upshape}
\renewcommand{\cftsecpagefont}{\rmfamily\mdseries\upshape} % No bold!

%%% END Article customizations

%%% The "real" document content comes below...

\title{BSc Project Proposal}
\author{El-Hassan Bilal Makled\\Media Engineering and Technology\\ German University in Cairo\\ Cairo, Egypt}
%\date{} % Activate to display a given date or no date (if empty),
         % otherwise the current date is printed 

\begin{document}
\maketitle

\section*{JiuJutsu Training Kinect Application}

\begin{tabular}{@{} l l}
  Proposed By: & El-Hassan Bilal Makled \\
  Application Number: & 13-8448 \\
  Proposed To: & Assoc.\ Prof.\ Georg Jung
\end{tabular}

\vspace*{\baselineskip}

Motion and gesture input systems are spreading in software development during the past few years. Some examples could be Microsoft's Kinect, Nintendo's Wii, or Sony's Move. 
The spread of this application ranged to affect video games and fitness programs. However, none of the fitness programs focused on actual contact sports techniques. 
In this project, we will use Microsoft Kinect to create an application that will take a group of JiuJutsu (a form of martial arts) moves as an input during a practice session, 
and compares them to a library of moves that are used as reference to calculate the delta. 
The project will also focus on measuring the impact of certain striking moves in different 
methods either by sound or by a sensor equiped Thai-pad depending on the situation of the practice session being captured.  

\subsection*{Tasks}
\begin{itemize}
\item Creating the libarary for the moves to be used as a reference.
\item Measuring and calculating the delta between the two moves, reference and captured move or previous user moves.
\end{itemize}

\end{document}
