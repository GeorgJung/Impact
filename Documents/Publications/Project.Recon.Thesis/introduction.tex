\chapter{Introduction}
\label{chap:intro}
With the current rise of ubiquitous computing touchless input sensor devices have emerged to replace controllers in some video games, house hold appliances or lately, mobile phones. With these devices present, programmers take advantage of developing applications either for entertainment, such as dancing games, sword fighting games or shooting games, or fitness related applications, like yoga.  

When it comes to fitness related applications, users are able to use them as support for keeping track of their performance. Some applications target to replace a coach, such as Nike+ Kinect training.\footnote{http://www.nike.com/us/en\_us/c/training/nike-plus-kinect-training} This program guides the user through a series of exercises including sit ups, push ups and more. The user in the practice is able to recieve real time feedback and coaching.

All the emerging fitness applications mostly target activities that could be done at a limited space such as a living room. Only few applications target contact sports such as Ju-Jutsu, Muay Thai, or Boxing. However, there applications available using touchless input sensors that target these types of sports, but only focuses on the gaming experience and do not act as fitness applications.

Ju-Jutsu, Muay Thai, Karate, Boxing, and other contact sports have a certain practice that requires a practitioner and a coach, where the coach is holding two Thai-Pads, a thick pad that covers the arms of the coach so that the trainee would punch or kick for practice. Usually this practice is fast paced with the coach always signalling for a certain move to be executed by the practitioner. With the practice being fast paced the practitioners sometimes find it hard to recognize their mistakes or evaluate their performance.

Project Recon focuses on creating a tool for practitioners to evaluate and support in keeping track of their performance. Project Recon is part of a larger project titled Impact. Project Impact is a fitness monitor project that targets Ju-Jutsu practitioners. It has multiple plugins for applications. Those applications take input from sensors and process them. This information is later sent to Project Impact which will visualize the information to the user. Their are various types of information about the practitioner. All of which later reflect the performance and work out results of the user. The applications used as plug-ins to Project Impact include Project Recon. Which uses a Microsoft Kinect sensor to locate the practitioner and recognizes his techniques. Other possible sensors are the sensor equipped Thai-Pads and a heart rate monitor.

%\section{Section Name} \label{sec:s1}
%Some sample text with an \ac{ac}, some citation \cite{citeKey1}, and some more \ac{ac2}.

%\section{Another Section}
%Reference to Section \ref{sec:s1}, and reuse of \ac{ac} nad \ac{ac2} with also full use of \acf{ac2}.
