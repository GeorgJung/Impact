\chapter{Features}\label{chap:features}

\section{Generic plugins}
One of the most important features of the monitor is the fact that it can be connected to any type of sensors. The monitor is built to accept any types of plugins, all what is needed is a couple of steps of installation \ref{sec:installation}. The aim of this feature is to open the door to future users and developers to expand the project for future plugins and sensors.

\section{Live session monitoring}
In this feature, the user creates a session (with a number of rounds, round duration and break duration) and then connects the sensor(s) needed and can do off-session measurements. Then the user starts the session and records the actions done.

One of the biggest challenges faced during the implementation of this feature was that the rounds should be saved to the database very quickly that the user would not notice that their is something happening in the background of the application. A proposed solution was that the round is saved at the end of it, this caused a small lag in the timer (this was very noticeable in sessions that had a short break between the rounds). The solution implemented was creating and saving the rounds objects to the database while creating the session and then when something is needed to be updated in the round's data, it is saved during the round, therefore minimizing the number of queries executed during the session.

\section{Statistics}

This feature serves to show the practitioners their progress over time. They can see their best-done moves, frequent-done moves, moves they need to practice on, rate of their moves with respect to the round's duration and other session-related info.

\section{Social}

In this feature, the practitioners can check other practitioners' level, progress and session data. The main reason behind this is to give the application a bit of gaming taste.