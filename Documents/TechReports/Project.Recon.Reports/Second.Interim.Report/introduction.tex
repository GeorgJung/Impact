\chapter{Introduction}
\label{chap:intro}
With the current rise of ubiquitous computing, motion sensor devices have emerged to replace controllers in some video games, house hold appliances or lately, mobile phones. With these devices present, people take advantage of using it to create applications either for entertainment, such as dancing games, sword fighting games or shooting games, or fitness related applications, like yoga.  
\\
\\
When it comes to fitness related applications, users are able to keep track of their performance. Some applications target to replace a coach, like Nike+ Kinect training. This program guides the user through a series of exercises, set ups, push ups and etc., where they would be able to get real time feedback and coaching.
\\
\\
All the rising fitness applications mostly targeted activities that could be done at home. Very little applications targeted contact sports like Ju-Jutsu, Muay Thai or Boxing. However, those type of sports using the motion sensors are already available, only focusing on the gaming experience and do not act as a fitness monitor.
\\
\\
This project focuses more on creating a tool for practitioners to evaluate their performance. Project Recon is part of a bigger project titled Impact. Project Impact is a fitness project that targets Ju-Jutsu practitioners. It has multiple sensors that collect different information about the practitioner and from this information would give results about his performance and work out. One of the sensors is Project Recon, which uses a Microsoft Kinect sensor to locate the practitioner and recognizes his techniques. Other possible sensors are the Thai-Pads and a heart rate monitor.

%\section{Section Name} \label{sec:s1}
%Some sample text with an \ac{ac}, some citation \cite{citeKey1}, and some more \ac{ac2}.

%\section{Another Section}
%Reference to Section \ref{sec:s1}, and reuse of \ac{ac} nad \ac{ac2} with also full use of \acf{ac2}.
